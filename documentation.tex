\documentclass{article}

\usepackage[utf8]{inputenc}
\usepackage[T1]{fontenc}

% load symbol definitions
\usepackage{textcomp}
% auto escape
\usepackage{underscore}

\usepackage[left=2.5cm,right=2.5cm,top=1cm,bottom=2cm]{geometry}

% code
\usepackage{listings}

\title{SBehavEd language documentation}
\author{Nils Laurent, 2A Apprentice at Ensimag}
\date{\today}

\begin{document}
\maketitle

\section{Files}
Use two files :\\
\texttt{<file_name>.txt} containing BehavEd code\\
\texttt{<file_name>.sb} containing SBehavEd code\\
CAUTION : \texttt{file_name} must be indentical for the \texttt{.txt} and \texttt{.sb} files. This is how the compiler associate the BehavEd file to the SBehavEd file.

\section{Values}
Regexp definition of values
\subsection{Identifier}
\texttt{[a-zA-Z]+[a-zA-Z0-9]*}
\subsection{String}
\texttt{"[a-zA-Z0-9]*"}
\subsection{Number}
\texttt{[0-9]+}


\section{Variables}
\subsection{Strings}
\begin{lstlisting}
variable_name = "string"
\end{lstlisting}
\subsection{Arrays}
\begin{lstlisting}
variable_name = ["array", "of", "strings"]
\end{lstlisting}

\newpage
\section{Identify BehavEd code and positions}
\subsection{Identify BehavEd code}
Use of \texttt{<<identifier>>} in comments surrounding BehavEd source code
\subsection{Identify positions in code}
Use of \texttt{<<@identifier>>} to create a label
\subsection{Example}
Copy animation of the "jawa" entity to the "rax" entity\\
In the BehavEd source file \texttt{source.txt} :
\begin{lstlisting}
affect ( "jawa", /*@AFFECT_TYPE*/ FLUSH )
{

	rem ( "<<jawa_dance>>" );

	task ( "back" )
	{
		set ( "SET_ANIM_BOTH", "BOTH_ATTACK_BACK" );
	}


	loop ( -1 )
	{

		loop ( 1 )
		{
			do ( "back" );
			wait ( 300.000 );
		}

	}

	rem ( "<<jawa_dance>>" );

}

affect ( "rax", /*@AFFECT_TYPE*/ FLUSH )
{

	rem ( "<<@rax_affect>>" );

}
\end{lstlisting}
In the SBehavEd source file \texttt{source.sb} :
\begin{lstlisting}
behaved_factor_code(rax_affect, jawa_dance)
\end{lstlisting}


\newpage
\section{Functions}

\subsection{\texttt{caffect_multiple}}
\subsubsection*{prototype}
\begin{lstlisting}
caffect_multiple(name_list, affect_type, destination_label)
\end{lstlisting}
\begin{lstlisting}
caffect_multiple(name, affect_type, destination_label)
\end{lstlisting}
\subsubsection*{argument type}
\texttt{name : String}\\
\texttt{name_list : Array of strings}\\
\texttt{affect_type : Identifier}\\
\texttt{destination_label : Identifier}

\subsection{\texttt{list_str}}
\subsubsection*{prototype}
\begin{lstlisting}
list_str(name, start_number, end_number)
\end{lstlisting}
\subsubsection*{argument type}
\texttt{name : String}\\
\texttt{start_number : Number}\\
\texttt{end_number : Number}
\subsubsection*{return value}
\texttt{RETURN VALUE : Array of strings}

\subsection{\texttt{behaved_factor_code}}
\subsubsection*{prototype}
\begin{lstlisting}
behaved_factor_code(destination_label, code_identifier)
\end{lstlisting}
\subsubsection*{argument type}
\texttt{destination_label : Identifier}\\
\texttt{code_identifier : Identifier}


\newpage
\section{Compile}
\subsection{On Windows}
\texttt{python.exe SBehavEdC.py <source_file>.sb}
\end{document}